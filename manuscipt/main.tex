% !TeX spellcheck = en_US
\documentclass[aps,twocolumn,pra,notitlepage,]{revtex4-2}
%twocolumn, 
\usepackage{amsmath,amssymb}
\usepackage{dsfont}
\usepackage{graphicx}
\usepackage{physics}
\usepackage{hyperref}
\usepackage{mathtools}
\usepackage{cancel}
\usepackage{bbold}
\usepackage{bm}
\usepackage{color} 
\usepackage[dvipsnames]{xcolor}
\usepackage[T1]{fontenc}

\usepackage{tcolorbox}

\usepackage{soul}
\newcommand{\ttau}{\boldsymbol{\tau}}

\newcommand{\kb}{k_\text{B}}
\newcommand{\ud}{\uparrow\downarrow}
\newcommand{\zem}{\omega}
\newcommand{\tauinf}{\tau}
\newcommand{\dzem}{\delta\zem}
\newcommand{\azem}{\overline\zem}
\newcommand{\tun}{t_{c}}
\newcommand{\te}{\tau}  % what to use for time? t or \tau?
\newcommand{\cnoise}{\xi}
\newcommand{\cnoisee}{\xi_\epsilon}
\newcommand{\cnoiset}{\xi_t}
\newcommand{\mnoise}{\eta}
\newcommand{\eadiab}{A}
\newcommand{\kv}{\mathbf k}
\newcommand{\vecsig}{\boldsymbol{\sigma}}
\newcommand{\JK}[1]{\textcolor{blue}{\small JK: #1}}
\newcommand{\LC}[1]{\textcolor{magenta}{#1}}
\newcommand{\adiab}[1]{{\tilde #1}}

\usepackage{ulem} % Crossing out with \sout{...}                      
\newcommand{\out}[1]{\textcolor{blue}{\sout{#1}}}
\newcommand{\add}[1]{\textcolor{red}{#1}}  


\begin{document}
\newcounter{theo}
\author{Jan A. Krzywda}\email{j.a.krzywda@liacs.leidenuniv.nl}
\author{Evert van Nieuwenburg}
\affiliation{$\langle aQa^L
\rangle$ Applied Quantum Algorithms, Lorentz Institute and Leiden Institute of Advanced Computer Science,
Leiden University, P.O. Box 9506, 2300 RA Leiden, The Netherlands}

\title{Reinformcement Learning for Efficient Resource Allocation Between Bayesian Estimation and Operation in Quantum Computer Limited by Low-Frequency Noise} 


\begin{abstract}

\end{abstract}
\maketitle

\section{Introduction}
\textbf{Intro} 
[] Coupling between the qubit and its environment is difficult to avoid. It result are the fluctuations of qubit Hamiltonian parameters, which detoriate its control and cause decoherence mechanisms. The effect of the noise strongly depends on its dynamics, in particulair the timescale on which it changes. While fast fluctuations of the environment are relevant for qubit relaxation (set by $T_1$ time), and uncorrelated dephasing ($T_2$ time), additional contribution to dephasing comes from the slow fluctuations of the environment. Related timescale $T_2^*$ is often the limiting factor of the operation of many solid state qubits.

[\textbf{Mitigation techniques}] The $T_2$-like dephasing due to uncorrelated noise, gives rise to phase flip channel, which in principle can be mittigated by Quantum Error Correction protocols \cite{}. More practically for near-term devices, similair goal can be achived using Quantum Error Mitigation techniques \cite{}, at the cost of measurments overhead. To achive similair goal in $T_2^*$ limited devices, most common approach is to decouple the qubit from the environment using dynamical decoupling \cite{}, or apply QEM after decorelating the noise through Pauli Twirling \cite{}. However, these methods are not always efficient, and can be costly in terms of resources. 

[\textbf{Bayesian tracking}]
Alternative approach to mittigate effects of slowely fluctuating paramters is to track and correct them in real-time. Tracking can be realised in Bayesian approach, in which the knowladge about Hamiltonian parameters is updated from the single shot of experiment with known probability distribution \cite{}. For constant and unknown parameters, the optimal estimation methods of Hamiltonian learning has been developed \cite{} and implemented \cite{}. However tracking time-dependent parameters poses additional constraints, as the estimation time cannot be longer than typical timescale at which the estimated paramter remains constant. This presents a trade-off between larger number of estimation shots (better precision) and worse accuracy due to changes in estimated parameter during estimation period. 

[\textbf{Strategy}]
Additional challange comes from the fact that the resources used for tracking the noise are decreasing shot budget for the qubit operation, and increases shot noise in the computation 

[\textbf{Spin qubits}] In spin qubits the relaxation time is orders of magnitude larger than dephasing time, due to small dipole moment of the two spin states \cite{}. For the dephasing the main contribution comes from the fluctuations of magnetic and electric fields, that are used to manipulate the qubit. 

[\textbf{Strategies}
]


[\textbf{RL for qubit control}] In this work we propose a Reinforcement Learning (RL) approach to mitigate the effect of the temporairly correlated noise on the qubit operation. The main idea is to use the slow fluctuations of the environment for learning, while the fast fluctuations are crucial for the qubit operation. The learning is done by the agent, that is responsible for the qubit control, and is based on the Bayesian estimation of the noise. The agent is trained to allocate the resources between the Bayesian estimation and the qubit operation, in order to maximize the qubit performance. The performance is measured by the qubit fidelity, which is the probability of the qubit to be in the desired state. 

[\textbf{Spin qubits}] In spin qubits the relaxation time is orders of magnitude larger than dephasing time, due to small dipole moment of the two spin states \cite{}. For the dephasing the main contribution comes from the fluctuations of magnetic and electric fields, that are used to manipulate the qubit. 

[\textbf{Strategies}
]
Recently experimental demonstration of noise tracking for Brown \cite{} and Pink \cite{} noise has been demonstrated. 

[\textbf{RL for qubit control}] In this work we propose a Reinforcement Learning (RL) approach to mitigate the effect of the temporairly correlated noise on the qubit operation. The main idea is to use the slow fluctuations of the environment for learning, while the fast fluctuations are crucial for the qubit operation. The learning is done by the agent, that is responsible for the qubit control, and is based on the Bayesian estimation of the noise. The agent is trained to allocate the resources between the Bayesian estimation and the qubit operation, in order to maximize the qubit performance. The performance is measured by the qubit fidelity, which is the probability of the qubit to be in the desired state. 


[Spin qubits] In spin qubits the relaxation time is orders of magnitude larger than dephasing time, due to small dipole moment of the two spin states \cite{}. For the dephasing the main contribution comes from the fluctuations of magnetic and electric fields, that are used to manipulate the qubit. 






\end{document}