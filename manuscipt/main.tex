% !TeX spellcheck = en_US
\documentclass[aps,twocolumn,pra,notitlepage,]{revtex4-2}
%twocolumn, 
\usepackage{amsmath,amssymb}
\usepackage{dsfont}
\usepackage{graphicx}
\usepackage{physics}
\usepackage{hyperref}
\usepackage{mathtools}
\usepackage{cancel}
\usepackage{bbold}
\usepackage{bm}
\usepackage{color} 
\usepackage[dvipsnames]{xcolor}
\usepackage[T1]{fontenc}

\usepackage{tcolorbox}

\usepackage{soul}
\newcommand{\ttau}{\boldsymbol{\tau}}

\newcommand{\kb}{k_\text{B}}
\newcommand{\ud}{\uparrow\downarrow}
\newcommand{\zem}{\omega}
\newcommand{\tauinf}{\tau}
\newcommand{\dzem}{\delta\zem}
\newcommand{\azem}{\overline\zem}
\newcommand{\tun}{t_{c}}
\newcommand{\te}{\tau}  % what to use for time? t or \tau?
\newcommand{\cnoise}{\xi}
\newcommand{\cnoisee}{\xi_\epsilon}
\newcommand{\cnoiset}{\xi_t}
\newcommand{\mnoise}{\eta}
\newcommand{\eadiab}{A}
\newcommand{\kv}{\mathbf k}
\newcommand{\vecsig}{\boldsymbol{\sigma}}
\newcommand{\JK}[1]{\textcolor{blue}{\small JK: #1}}
\newcommand{\LC}[1]{\textcolor{magenta}{#1}}
\newcommand{\adiab}[1]{{\tilde #1}}

\usepackage{ulem} % Crossing out with \sout{...}                      
\newcommand{\out}[1]{\textcolor{blue}{\sout{#1}}}
\newcommand{\add}[1]{\textcolor{red}{#1}}  


\begin{document}
\newcounter{theo}
\author{Jan A. Krzywda}\email{j.a.krzywda@liacs.leidenuniv.nl}
\author{Evert van Nieuwenburg}
\affiliation{$\langle aQa^L
\rangle$ Applied Quantum Algorithms, Lorentz Institute and Leiden Institute of Advanced Computer Science,
Leiden University, P.O. Box 9506, 2300 RA Leiden, The Netherlands}

\title{Reinformcement Learning for Efficient Resource Allocation Between Bayesian Estimation and Operation in Quantum Computer Limited by Low-Frequency Noise} 


\begin{abstract}

\end{abstract}
\maketitle

\section{Introduction}
\textbf{Intro} Uncontrolled drift of Hamiltonian parameters results in temporally correlated noise that affects quantum computation. It 

Such drift can be modeled as a unknown term in the Hamiltonian, which if averaged over realisations give rise to a decoherence. In many cases the noise is temporairly correlated, which means that the noise at time $t$ is correlated with the noise at time $t+\tau$. 



With temporairly correlated noise, fast fluctuations are crucial for the qubit operation, while slow fluctuations can be used for learning.



\end{document}